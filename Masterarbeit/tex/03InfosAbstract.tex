\chapter*{Common Information}
\thispagestyle{plain}
\pagestyle{plain}

\begin{tabular}{p{0.3\textwidth}p{0.65\textwidth}}

Name: & \Author \\*[0.2cm]
Institution: & Salzburg University of Applied Sciences \\*[0.2cm]
Degree Programme: & Information Technology \& Systems Management \\*[0.2cm]
Title of Thesis: & \Title \\*[0.2cm]
Keywords: & \Keywords  \\*[0.2cm]
Supervisor: & \Advisor

\end{tabular}

\section*{\Large\bfseries Abstract}



Protein homology classification is an important task to better understand proteins whose three-dimensional structure and function are not obvious. Current methods that rely on the primary structure of a protein do not always find distant homologous relationships between proteins.  
%
This thesis examines the usage of secondary structure information in profile Hidden Markov Models to improve the classification accuracy in protein family prediction.
%
This is done by extending the emission frequencies of the primary structure by secondary structure frequencies. A generalized mixed frequency set is generated using optimized weighting techniques. 
The secondary structure is determined by the three-dimensional structure, if available, or predicted from the primary structure.
%
To assess the effectiveness of the different weighting methods, the implementation has been tested with 69 selected sequence alignments, representing distant related families. These sequences have been scored against the SCOP database to determine the accuracy of finding distant homologous relationships between proteins. 
%
Results show that the integration of secondary structure information improves the accuracy of homology prediction.  